\documentclass{article}
\usepackage{hyperref}

\title{Image Analysis Application Documentation}
\author{}
\date{}

\begin{document}

\maketitle

\section{Introduction}
This application takes an image URL, analyses the image, and outputs the image analysis. The image analysis is performed using the OpenAI API Vision.

For more information about the model, visit the \href{https://platform.openai.com/docs/guides/vision}{OpenAI Vision Documentation}.

\section{Usage}

\begin{enumerate}
    \item Install required node modules by running \texttt{npm install} in the console.
    \item \textbf{Required:} A \texttt{.env} file with \texttt{PORT} and \texttt{OPENAI\_API\_KEY} values. (For the purpose of this homework, a temporary key is provided in \texttt{server.js}.)
    \item To launch the application on localhost, use \texttt{npm start}.
    \item Navigate to \url{http://localhost:3000/}
\end{enumerate}

On the webpage, enter the image URL and click "Analyze Image." You will then see the text description of the image.

\section{Sample Images}

For sample images, use the following URLs:

\begin{itemize}
    \item \url{https://picsum.photos/id/1/200/300.jpg}
    \item \url{https://picsum.photos/id/2/200/300.jpg}
    \item \url{https://picsum.photos/id/3/200/300.jpg}
    \item \url{https://picsum.photos/id/4/200/300.jpg}
\end{itemize}

Users may change the number after \texttt{id} for more images.

\end{document}
